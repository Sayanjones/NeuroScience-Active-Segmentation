% !TeX spellcheck = en_US
\documentclass{amsart}
\usepackage{amsaddr}
 
%%%%%%%%%%%%%%%
%  Help file for the AS\IJ platform
%%%%%%%%%%%%%%% 
\title{Active Segmentation Moments and Texture Features}

\author{Dimiter Prodanov\textsuperscript{1,2} }
\address{
	\textsuperscript{1}NERF, IMEC, Kapeldreef 75, 3001 Leuven, Belgium; 
	\textsuperscript{2}PAML-LN,
	IICT, Bulgarian Academy of Sciences,
	Sofia, Bulgaria
}
%\email{dimiter.prodanov@imec.be; dimiterpp@gmail.com}
%%%%%%%%%%%
% Doc
%%%%%%%%%%%
\begin{document}
\date{\today}	
\maketitle
\tableofcontents

%%%%%%%%%%%%
% Section
%%%%%%%%%%%%
 %%%%%%%%%%%%%
 %  Section
 %%%%%%%%%%%%%
 \section{Image Moments}\label{sec:immom}
 
 Spatial and central moments are important statistical properties of an image. 
 Mathematically, the image moment is generally defined as the inner product  of the image intensity function f(x,y) and a certain basis function $ P_{m,n}$.  
 In the continuous approximation, the moments of a function are computed by the integral
 \[
 M_{m,n}   = \iint\limits_{I} P_{m,n}(x,y) f(x,y) dx dy
 \]
 where $ P_{m,n}(x,y)$ is polynomial, parameterized by the integers \textit{m} and \textit{n}.   
  depending on whether the basis functions satisfy orthogonality, the image moments can
 be classified into orthogonal moments and non-orthogonal moments.
 
 %%%%%%%%%%%%%%
 %  Sec
 %%%%%%%%%%%%%%
 \section{Raw and Central Moments}\label{sec:rawmom}
 For example, the raw image moments are given by the homogeneous form $P_{m,n}(x,y)= x^m y^n$.
 The moments, can be referred to the center of the image frame or to the center of mass of the image $(x_c, y_c)$, 
 in which case,   $P_{m,n}(x,y) = (x-x_c)^m (y-y_c)^n$.
 The two main problems with such a choice is that the moments contain redundant information because the homogeneous polynomials are not orthogonal; 
 also the computation loses numerical precision due to cancellation of large terms. 
 Mathematically, a better choice of  polynomials is a polynomial from an orthogonal  family. 
 Such polynomials enjoy an expansion property, that is 
 \[
 f(x,y) = \sum_{m=0}^{\infty} \sum_{n=0}^{\infty} M_{m,n} P_{m,n}(x,y)
 \]
 Useful examples of such orthogonal families are the Legendre and Zernike polynomials.

%%%%%%%%%%%%
%  Sec
%%%%%%%%%%%%
\section{Legendre moments}\label{sec:legendre}
 	The Legendre polynomials form an orthogonal set on the interval $[-1, 1]$ 
 	%and can be defined by the Rodriguez formula
 	%\begin{equation}
 	%L_{n} (x) = \frac{1}{2^n n!} \left( \frac{d}{dx} \right)^n (x^2-1)^n
 	%\end{equation}
 	The Legendre polynomials enjoy a two term recurrence relation %\footnote{\url{https://mathworld.wolfram.com/LegendrePolynomial.html}}
 	\begin{equation}
 		(n+1)L_{n+1} (x)=  (2 n + 1) x L_{n} (x) - n L_{n-1}(x) , \quad L_0(x)=1, \quad  L_1(x)= x
 	\end{equation}
 	that was used for their computation in the present paper. 
 	The advantage of the recursion relation is that all Legendre moments up to a user-defined order can be computed simultaneously. 

%%%%%%%%%%%%
%  Section
%%%%%%%%%%%%
\section{Zernike moments} \label{sec:zernike}
 	The Zernike polynomials are normalized on the unit disk in the complex plane.
 	The radial  Zernike polynomials   can be defined for $n-m$ even as:
 	\begin{equation}
 		R_{n}^{m} (r) = \sum_{l=0}^{ (n-m)/2 } \frac{(-)^l  (n-l)! }{l! ( (n+m)/2 -l )! ( (n-m)/2 -l)! } r^{n - 2 l}
 	\end{equation}
 	and 0 otherwise.
 	The present paper implemented a  recursive computation method given by the formula \cite{Shakibaei2013}
 	\begin{equation}
 		R^m_n(r)=r \left( R^{|m-1|}_{n-1} (r) +  R^{m+1}_{n-1} (r) \right)  - R^{m}_{n-2} (r), \quad R^0_0=1
 	\end{equation}
 	The orthogonal Zernike polynomials then are
 	\begin{equation}
 		V_{mn}(r, \theta):= R_{n}^{m} (r) e^{-i m \theta}
 	\end{equation}
 	The normalization of the polynomials for grayscale images is not an issue because they have a fixed dynamic range, 
 	so an image can be always normalized to unit range prior to computation of an image moment.  

%%%%%%%%%%%
%  Section
%%%%%%%%%%
\section{Haralick features}

Haralick features \cite{Haralick1973} are coded as following
0 --  Angular 2\textsuperscript{nd} Moment;
1 -- Contrast;
2 -- Correlation;
3 -- Dissimilarity;
4 -- Energy;
5 -- Entropy;
6 -- Homogeneity.

%%%%%%%%%%
% Section
%%%%%%%%%%
\section{ImageJ statistics}\label{sec:istat}
The following ImageJ statistics are computed:
Area, mean, stdev, min, max, centroid, center of mass,  perimeter, ellipse, shape descriptors, Ferret's diameter, integrated density, median, skewness, kurtosis, area fraction.

%AREA+MEAN+STD_DEV+MODE+MIN_MAX+
%CENTROID+CENTER_OF_MASS+PERIMETER+RECT+
%ELLIPSE+SHAPE_DESCRIPTORS+FERET+INTEGRATED_DENSITY+
%MEDIAN+SKEWNESS+KURTOSIS+AREA_FRACTION;

%%%%%%%%%%%%%%%%%%%%%%
%  Bib
%%%%%%%%%%%%%%%%%
\bibliographystyle{plain}
\bibliography{wekasegmentation}
\end{document}